% \documentclass[a4paper]{amsart}
\documentclass[20pt, a4paper]{extarticle}
\usepackage[]{graphicx}
\usepackage{amsmath}
\usepackage{extsizes}
\date{}

\setlength{\parindent}{0.0in}
\setlength{\parskip}{0.1in}

\newcommand{\laplace}[1]{\mathcal{L}\{#1\}}
\newcommand{\Hv}{\textrm{H}}
\renewcommand{\b}{\mathbf}

\begin{document}
\title{6G5Z3011 Multi-variable calculus and analytical methods}
\author{Tutorial Sheet 04}
\maketitle

Qs 1 -- 4 on \textbf{Line integrals}
\begin{enumerate}
    \item
    A, B and D are the points $(0,0)$, $(2,0)$ and $(2,1)$ respectively. Evaluate the path integral
    $$\int\limits_C (x^2 + 2y + 4) \, dx + (x^2 + 2y +4) \, dy$$
    when (a) $C$ is the straight line segment AD and (b) when $C$ is the path made up of the straight line segments AB and BD.
    \item
    Evaluate the path integral
    $$\int\limits_C (x^2 + 2y) \, dx + (x + y^2) \, dy $$
    where $C$ is the segment of the line $y=2x+1$ from $(1,3)$ to $(3,7)$.
    \item
    Evaluate the path integral
    $$\int\limits_C x \, dy + (y + 1) \, dx $$
    where $C$ is
    \begin{enumerate}
    \item
    the segment of the curve $y=\sin x$ from $(0,0)$ to $(\frac{\pi}{2},1)$,
    \item
    the segment of the line $y=\frac{2x}{\pi}$ from $(0,0)$ to $(\frac{\pi}{2},1)$,
    \item
    any other path from $(0,0)$ to $(\frac{\pi}{2},1)$.
    \end{enumerate}
    \item
    When a force $\b{F}$ moves along a path $C$ in the plane then the total workis given by
    $$\int\limits_C \b{F} . \, d \b{r}$$
    where $\b{r}(x,y)$ is the position vector $x \b{i} + y \b{j}$.
    
    Show that when the force $\b{F}$, given by $\b{F}(x,y) = xy \b{i} + y^2 \b{j}$, moves along the path $C$, defined by $t \b{i} + t^2 \b{j}$ where $0 \leq t \leq 1$, the work done by the force is $\frac{7}{12}$.
    \end{enumerate}


\end{document}