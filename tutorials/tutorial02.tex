\documentclass[a4paper]{amsart}

\usepackage[]{graphicx}

\setlength{\parindent}{0.0in}
\setlength{\parskip}{0.1in}

\newcommand{\laplace}[1]{\mathcal{L}\{#1\}}
\newcommand{\Hv}{\textrm{H}}

\begin{document}
\title{6G5Z3011 Multi-variable calculus and analytical methods}
\author{Tutorial Sheet 02}
\maketitle

\textbf{Small increments formula questions}
\begin{enumerate}

\item
Use the small increments formula to estimate the change in the function
$$f(x,y)=\left ( x+2y^2 \right )^5$$
when $x$ and $y$ both increase from 1 to 1.01. Check your answer by direct evaluation of the values of $f$.
\item
The pressure exerted by a column of gas of density $D$ and length $L$ is $P=\frac{1}{3} DL^2$. Find the percentage change in pressure caused by a 1\% increase in density and a 2\% increase in length.
\item
The surface area $S$ and the volume $V$ of a right circular cone of base radius $r$ and height $h$ are given by 
\begin{align*}
S&=\pi r^2 + \pi r \left ( r^2 + h^2 \right )^{\frac{1}{2}} \\
V&=\frac{1}{3} \pi r^2 h
\end{align*}
\begin{enumerate}
\item
What percentage increase in $h$ will cancel out a 2\% decrease in $r$ if the volume is to required to remain the same. 
\item
In an attempt to estimate the value of $S$ the radius $r$ is measured as $5.00 \, \pm \, 0.01 \text{mm}$ and the height $h$ as  $12.00 \, \pm \, 0.01 \text{mm}$. Estimate the value of $S$ and quantify the possible error.

If it is possible to measure just one of the quantities $h$ and $r$ more accurately, which one would you choose? Justify your answer.
\end{enumerate}
\item
Consider the function $f$ of $n$ variables defined by
$$f(x_1,x_2, \dots , x_n) = x_1^{p_1} x_2^{p_2} \dots x_n^{p_n} .$$
Suppose that for each $r$ there is an $i_r$\% increase in the value of $x_r$. Show that the total increase in the value of $f$ is $p_1 i_1 + p_2 i_2 + \dots + p_n i_n$.

\end{enumerate}
\textbf{Chain rule and Jacobian questions}

\begin{enumerate}
    \setcounter{enumi}{4}
\item




Show that if the Cartesian  coordinates $(x,y)$ are transformed to the polar coordinates $(r,\theta)$ and $f(x,y)$ is a differentiable function of $x$and $y$ then 
$$x \frac{\partial f}{\partial y} - y \frac{\partial f}{\partial x} = \frac{\partial f}{\partial \theta}.$$
\item
Consider the transformation from the coordinates $(x,y)$ to a new pair $(s,t)$ defined by 
$$s=xy, \quad t=\frac{1}{y} .$$
If $f(x,y)$ is a differentiable function of $x$ and $y$ show that 
$$y \frac{\partial f}{\partial y} \left ( x \frac{\partial f}{\partial x} - y \frac{\partial f}{\partial y} \right ) = t \frac{\partial f}{\partial t} \left ( s \frac{\partial f}{\partial s} - t \frac{\partial f}{\partial t} \right ) .$$
\item
Show that the Jacobian of the transformation from Cartesian coordinates $(x,y)$ to polar coordinates $(r,\theta)$ is given by 
$$\frac{\partial (x,y)}{\partial (r, \theta)} = r.$$
\item
Use the chain rule to prove that if a coordinate transformation from $(x,y)$ to $(s,t)$ is composed with one from $(s,t)$ to $(u,v)$ then 
$$\frac{\partial (x,y)}{\partial (s, t)} \frac{\partial (s,t)}{\partial (u, v)} = \frac{\partial (x,y)}{\partial (u,v)}.$$
Hence establish the reciprocal rule that if a transformation from $(x,y)$ to $(s,t)$ is composed with the inverse transformation from $(s,t)$ to $(x,y)$ then 
$$\frac{\partial (x,y)}{\partial (s, t)} \frac{\partial (s,t)}{\partial (x, y)} = 1.$$
\item
Show that the Jacobian of the transformation from polar coordinates $(r,\theta)$ to Cartesian coordinates $(x,y)$ is given by 
$$\frac{\partial (r,\theta)}{\partial (x, y)} = \frac{1}{r}.$$
\item
The Cartesian coordinates $(x,y)$ are transformed by the mapping 
$$ s = y-x, \quad t=(y-x)^2.$$
Complete the following table of coordinate values

\bigskip

\begin{tabular}{|c|c|c|c|c|c|c|c|c|c|c|c|c|c|c|c|c|}
\hline  $x$& 0 & 0 & 1 & 1 & 2 & 0 & 2 & 1 & 2 & 3 & 0 & 3 & 1 & 3 & 2 & 3 \\ 
\hline  $y$& 0 & 1 & 0 & 1 & 0 & 2 & 1 & 2 & 2 & 0 & 3 & 1 & 3 & 2 & 3 & 3 \\ 
\hline  $s$&  &  &  &  &  &  &  &  &  &  &  &  &  &  &  &  \\ 
\hline  $t$&  &  &  &  &  &  &  &  &  &  &  &  &  &  &  &  \\ 
\hline 
\end{tabular} 

\bigskip

Plot the resulting points $(s,t)$ from the table and comment on what you notice. Find the Jacobian $\frac{\partial (s,t)}{\partial (x, y)}.$ What can be said about the inverse transformation?
\end{enumerate}


\end{document}